\begin{abstract}
在计算机历史上,每一次技术的进化都伴随着数据的爆炸,这些海量增长的数据日益成为人们所关注的焦点。在计算机行业内,如何高效快速的处理这些海量数据,一直是学术界所热议的话题。

MapReduce是Google公司在2004年在OSDI国际会议上提出的一种简单的并行计算模型,它借鉴了函数式编程语言的特点,将大规模数据的分布式处理程序抽象为一个运行在分布式集群上的Map函数和Reduce函数,从而实现了分布式处理海量数据。

Hadoop是MapReduce的开源实现,本文基于Hadoop平台,针对Hadoop的两种运行机制设计并开发出了Web访问日志处理程序,并通过一定的测试将Hadoop两种运行机制的性能做以对比,继而找出Hadoop现有不足。

\keywords{MapReduce,Hadoop,开发,性能}
\end{abstract}

\begin{englishabstract}
Every evolution in the computer's history comes with the explosion of data. People has shown a lot of interests to the massive growth of data. How to process these data rapidly and efficiently has been a hot topic by the academics. 

MapReduce is a simple programing framework published by Google Company in 2004 in OSDI International Conference. It draws on the characteristics of functional programming languages​​ and abstracts a large-scale distributed data handler as a pair of Map and Reduce functions which run in a distributed cluster to process huge amounts of data. 

Hadoop is an open source implementation of MapReduce. This thesis based on Hadoop platform. Aim to develop a software, which can process the access logs generated by webserver, in two different ways provided by Hadoop and compare the performance of these two methods then found the insignificancy of Hadoop platform.

\englishkeywords{MapReduce, Hadoop, Programing, Performence}
\end{englishabstract}

