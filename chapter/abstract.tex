\begin{abstract}
MapReduce是Google公司于2004年在OSDI国际会议上提出的一种简单的并行计算模型,该模型将大规模数据的分布式处理程序抽象为一个运行在分布式集群上的Map函数和Reduce函数,从而实现分布式处理海量数据。Hadoop是该计算模型的一种开源实现。

基于Hadoop的MapReduce框架是目前海量数据处理的重点研究方向之一。本文对基于Hadoop的MapReduce框架运行机制和过程进行了深入的研究,使用两种不同的实现方式设计并开发出了Web访问日志分析程序,进一步验证了MapReduce框架的正确性。同时,本文通过对实验结果的分析,找出MapReduce框架潜在的瓶颈,给出了较为全面的结论和建议,对今后基于Hadoop的MapReduce搭建及编程有着一定指导性作用。


\keywords{MapReduce  Hadoop  程序开发  性能测试  性能瓶颈}
\end{abstract}

\begin{englishabstract}
MapReduce is a simple programing framework published by Google Company in 2004 in OSDI International Conference. It abstracts a large-scale distributed data handler as a pair of Map and Reduce functions which run in a distributed cluster to process huge amounts of data. Hadoop is an open source implementation of MapReduce.

Nowadays, the research in processing massive data based on Hadoop and MapReduce framework is a key direction. This thesis gives out an in-depth study of the mechanism and process of the Hadoop and MapReduce framework. The work in the thesis also included designing and developing a software, which can process the access logs generated by webserver, in two different ways provided by Hadoop. During the performance and correctness tests, thesis not only proves the MapReduce framework can process the data correctly, but also finds some potential performance bottlenecks of it. The conclusion and suggestion set out in the thesis may provide some guidance to the environment building and programing based on the Hadoop and MapReduce.

\englishkeywords{MapReduce Hadoop programing performence test      performance bottlenecks}
\end{englishabstract}

