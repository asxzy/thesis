\chapter{本文总结}
\label{chap:5}

本文通过对基于Hadoop实现的MapReduce编程框架的介绍,详述了MapReduce编程思想及Hadoop的实现过程,并在此基础上使用两种实现方式实现了Web访问日志程序的编写与其性能分析。

根据第三章的测试和第四章的分析,可以得出以下四个结论:

\begin{enumerate}
\item MapReduce运行效率高,是个切实可行的分布式编程框架。MapReduce分布式程序编写简单,开发难度低,利用Streaming和Pipes可以快捷的将原有软件进行移植。
\item Hadoop作为一款开源软件,从底层实现了MapReduce框架,具有一定的扩展性、容错容灾性,是一款成熟的MapReduce实现软件。
\item 节点机的本地I/O和网络传输可能会造成MapReduce性能瓶颈,在大集群中,这种瓶颈会变得尤为明显。何合理的设计集群,合理的对HDFS进行配置,对提高集群间效率有着不可忽视的作用。
\item 由于Streaming的Map结果合并部分处理不理想,导致程序I/O量较大且Reduce压力较大。因此现阶段,原生Java比Streaming效率高。
\item MapReduce程序编写中,Key的选择对数据合并、Reduce分发有着至关重要的作用。编程中,开发者可以利用数据的分布规律和一定的开发技巧将Key打散,以便更好地利用各分布节点。
\end{enumerate}