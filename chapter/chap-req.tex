\chapter[实验环境搭建]{实验环境搭建}
\label{chap:fuluA}
Hadoop使用Java语言实现,集群之间以ssh协议通信,故Hadoop的依赖较少,且安装较为简便,其大体过程为:安装操作系统,安装ssh和jdk,解压缩Hadoop二进制包,编辑Hadoop配置文件并设置ssh、jdk和hadoop环境变量。
本文以具体过程如下:

1.最小化安装centos\_6.2\_x86-64

2.设置网络,并将源设置为西电源,关闭防火墙

\begin{verbatim}
iptables -F
iptables -X
iptables -Z
iptables-save > /etc/sysconfig/iptables
\end{verbatim}
3.更新系统

\begin{verbatim}
yum upgrade –y
\end{verbatim}
4.安装依赖包rsync openssh-server openssh-clients

\begin{verbatim}
yum install -y rsync openssh-server openssh-clients
\end{verbatim}
5.下载sun-jdk和hadoop包
6.安装jdk到/usr/lib并创建相应软链接

\begin{verbatim}
cp jdk-6u31-linux-x64.bin /usr/lib
cd /usr/lib
sh jdk-6u31-linux-x64.bin
rm jdk-6u31-linux-x64.bin
ln -s jdk1.6.0_31 jdk
\end{verbatim}
7.创建hadoop用户和用户组,并设置密码

\begin{verbatim}
groupadd hadoop
useradd -g hadoop hadoop –d /home/hadoop
passwd hadoop
\end{verbatim}
8.创建ssh密钥公钥,并加入可信列表

\begin{verbatim}
su hadoop
ssh-keygen -t rsa -P ""
cat ~/.ssh/id_rsa.pub >> ~/.ssh/authorized_keys
\end{verbatim}
验证密钥登录是否生效

\begin{verbatim}
ssh localhost
\end{verbatim}
若没生效,检查文件权限

\begin{verbatim}
chmod 644 ~/.ssh/authorized_keys
\end{verbatim}
9.解压缩hadoop-1.0.1到/usr/local并创建软链接

\begin{verbatim}
su
cp hadoop-1.0.1.tar.gz /usr/local
cd /usr/local
tar zxvf hadoop-1.0.1.tar.gz
ln -s hadoop-1.0.1 hadoop
chown hadoop:hadoop hadoop-1.0.1 hadoop -R
\end{verbatim}
10.给hadoop用户设置环境变量

\begin{verbatim}
vi ~/.bashrc
\end{verbatim}
文件末尾添加环境变量如下:

\begin{verbatim}
# Set Hadoop-related environment variables
export HADOOP_HOME=/usr/local/hadoop
# Set JAVA_HOME (we will also configure JAVA_HOME directly for Hadoop later on)
export JAVA_HOME=/usr/lib/jdk
# Add Hadoop bin/ directory to PATH
export PATH=$PATH:$HADOOP_HOME/bin
\end{verbatim}
重新登录使变量生效
11.编辑hadoop配置(/usr/local/hadoop/conf)

\begin{verbatim}
conf/hadoop-env.sh
\end{verbatim}
替换

\begin{verbatim}
# export JAVA_HOME=/usr/lib/j2sdk1.5-sun
\end{verbatim}
为

\begin{verbatim}
export JAVA_HOME=/usr/lib/jdk
\end{verbatim}
添加

\begin{verbatim}
# Disable the HADOOP_HOME_WARN
export HADOOP_HOME_WARN_SUPPRESS=TRUE
\end{verbatim}
取消hadoophome设置警报
core-site.xml 

\begin{verbatim}
<configuration>
     <property>
         <name>fs.default.name</name>
         <value>hdfs://localhost:9000</value>
     </property>
</configuration>
\end{verbatim}
hdfs-site.xml 

\begin{verbatim}
<configuration>
     <property>
         <name>dfs.replication</name>
         <value>1</value>
     </property>
</configuration>
\end{verbatim}
mapred-site.xml 

\begin{verbatim}
<configuration>
     <property>
         <name>mapred.job.tracker</name>
         <value>localhost:9001</value>
     </property>
</configuration>
\end{verbatim}
12.格式化hdfs

\begin{verbatim}
hadoop namenode -format
\end{verbatim}
13.启动hadoop

\begin{verbatim}
start-all.sh
\end{verbatim}
运行命令jps,查看Hadoop进程启动情况。如果
TaskTracker、DataNode、Jps、NameNode、SecondaryNameNode和JobTracker子进程均成功启动,则Hadoop运行正常。
至此,Hadoop就安装完成了。



